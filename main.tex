\documentclass{beamer}


\usetheme{Frankfurt}  


\title{Komety}
\author{Damian Dominiak}
\date{05.02.2025}

\begin{document}

% Strona tytułowa
\begin{frame}
    \titlepage
\end{frame}


\begin{frame}{Czym są komety}
    \begin{itemize}
        \item Komety to ciała niebieskie, które składają się z pyłu, fragmentów skalnych i lodu. 

        \item Obiegają Słońce po bardzo wydłużonych orbitach, które mogą sięgać nawet setek tysięcy jednostek astronomicznych.

    \end{itemize}
\end{frame}


\begin{frame}{Budowa komet}
    \begin{enumerate}
        \item Jądro
        \item Ogon
        \item Koma kometarna
    \end{enumerate}
\end{frame}


\begin{frame}{Podział komet}
    \begin{enumerate}
        \item Komety krótkookresowe
        \item Komety długookresowe
    \end{enumerate}
\end{frame}

\begin{frame}{Ciekawostki}
    \begin{enumerate}
        \item Kiedyś komety były uważane za zwiastun ważnych wydarzeń na Ziemi 
        \item "kometa"-"kometes"-"włochata gwiazda"
        \item Najdłuższy znany ogon komety wynosił ponad 300 milionów kilometrów długości
        \item  Gdy kometa zbliża się do słońca mogą powstać z niej meteoryty

    \end{enumerate}
\end{frame}

\begin{frame}{Źródła}
    \begin{enumerate}
        \item https://pl.wikipedia.org/wiki/Kometa
        \item https://www.youtube.com/watch?v=TMOOqPbpsWY
        \item https://pl.wikipedia.org/wiki/Kometa_Halleya
        \item  https://pl.wikipedia.org/wiki/Kometa_kr%C3%B3tkookresowa


    \end{enumerate}
\end{frame}


\end{document}